
\documentclass[letterpaper,11pt]{article}
%%%%%%%%%%%%%%%%%%%%%%%%%%%%%%%%%%%%%%%%%%%%%%%%%%%%%%%%%%%%%%%%%%%%%%%%%%%%%%%%%%%%%%%%%%%%%%%%%%%%%%%%%%%%%%%%%%%%%%%%%%%%%%%%%%%%%%%%%%%%%%%%%%%%%%%%%%%%%%%%%%%%%%%%%%%%%%%%%%%%%%%%%%%%%%%%%%%%%%%%%%%%%%%%%%%%%%%%%%%%%%%%%%%%%%%%%%%%%%%%%%%%%%%%%%%%
\usepackage{graphicx}
\usepackage{amsmath,amsfonts,amssymb,amsthm,float}
\usepackage{hyperref}
\usepackage[utf8]{inputenc}
\usepackage[left=2cm, right=2cm, top=2cm, bottom=2cm]{geometry}

\setcounter{MaxMatrixCols}{10}
%TCIDATA{OutputFilter=LATEX.DLL}
%TCIDATA{Version=5.50.0.2953}
%TCIDATA{<META NAME="SaveForMode" CONTENT="1">}
%TCIDATA{BibliographyScheme=BibTeX}
%TCIDATA{LastRevised=Monday, May 19, 2025 15:27:46}
%TCIDATA{<META NAME="GraphicsSave" CONTENT="32">}

\input{tcilatex}
\renewcommand{\baselinestretch}{1.15}
\setlength{\parindent}{0pt}
\setlength{\parskip}{0.5\baselineskip}
\pretolerance=2000 \tolerance=3000
\renewcommand{\abstractname}{Resumen}

\begin{document}

\title{An\'{a}lisis de sistemas biol\'{o}gicos}
\author{Jesus Zamora Cervantes $\left[ 21212185\right] $ \\
%EndAName
Departamento de Ingenier\'{\i}a El\'{e}ctrica y Electr\'{o}nica\\
Tecnol\'{o}gico Nacional de M\'{e}xico / Instituto Tecnol\'{o}gico de Tijuana%
}
\maketitle

\noindent \textbf{Palabras clave: }Lotka-Volterra; C\'{e}lulas efectoras; L%
\'{o}gistico; Inmunoterapia; Lyapunov.

\noindent

\bigskip

\noindent Correo: \textbf{l21212185@tectijuana.edu.mx}

\noindent \noindent Carrera: \textbf{Ingenier\'{\i}a Biom\'{e}dica }

\noindent Asignatura: \textbf{Gemelos Digitales}

\noindent Profesor: \href{https://biomath.xyz/}{\textbf{Dr. Paul Antonio
Valle Trujillo}} (paul.valle@tectijuana.edu.mx)

\section{Modelo matem\'{a}tico}

El modelo matem\'{a}tico se compone por las siguientes tres Ecuaciones
Diferenciales Ordinarias (EDOs) de primer orden:

\begin{eqnarray*}
\dot{x} &=&r_{1}x(1-b_{1}x)-a_{12}xy-a_{13}xz, \\
\dot{y} &=&r_{2}y(1-b_{2}y)-a_{21}xy, \\
\dot{z} &=&(r_{3}-a_{31})xz-d_{3}z+\rho _{i,}
\end{eqnarray*}

donde $x\left( t\right) $ es la poblaci\'{o}n de c\'{e}lulas anormales, $%
y\left( t\right) $ la poblaci\'{o}n de c\'{e}lulas normales y $z\left(
t\right) $ la poblaci\'{o}n de c\'{e}lulas efectoras, adem\'{a}s el tiempo \ 
$t$ se mide en d\'{\i}as.

\begin{enumerate}
\item Tanto la ecuaci\'{o}n $\dot{x}$ como $\dot{y}$ est\'{a}n descritas por
la ley de crecimiento log\'{\i}stico con tasas de crecimiento definias por $%
r_{1}$ y $r_{2\text{ .}}$

\item La EDO de $\dot{z}$ describe el crecimiento de c\'{e}lulas efectoras
mediante la ley de acci\'{o}n de masas.

\item Los t\'{e}rminos $xy$ en las tres ecuaciones representan la
competencia de recursos entre c\'{e}lulas patol\'{o}gicas y c\'{e}lulas
sanas.

\item Los t\'{e}rminos $xz$ determinan la supresi\'{o}n inmune.

\item El par\'{a}metro de control $\rho _{i}$ representa la aplicaci\'{o}n
externa de un tratamiento de inmunoterapia.

\item La din\'{a}mica del sistema es de la forma presa-depredador de
Lotka-Volterra.

\item Debido que el sistema describe la concentraci\'{o}n de poblaciones
celulares con respecto al tiempo, su soluciones deben ser no negativas para
condiciones iniciales no negativas, de lo contrario se perder\'{\i}a el
significado biol\'{o}gico del sistema.
\end{enumerate}

\section{An\'{a}lisis de positividad}

En esta secci\'{o}n se aplica el lema de positividad para sistemas din\'{a}%
micos no lineales, por lo que se realizan las siguientes evaluaciones:

\begin{eqnarray}
\left. \dot{x}\right\vert _{x=0} &=&r_{1}(0)(1-b_{1}\left( 0\right)
)-a_{12}\left( 0\right) y-a_{13}\left( 0\right) z=0  \label{dx} \\
\left. \dot{y}\right\vert _{x=0} &=&r_{2}(0)(1-b_{2}\left( 0\right)
)-a_{21}x\left( 0\right) =0  \label{dy} \\
\left. \dot{z}\right\vert _{x=0} &=&(r_{3}-a_{31}\left( 0\right) x\left(
0\right) -d_{3}\left( 0\right) +\rho _{i}=\rho _{i}  \label{dz}
\end{eqnarray}

Por lo tanto de acuerdo con De Leenheer \& Aeyels [1], se concluye el
siguiente resultado:

\bigskip

Resultado I. Positividad: Las soluciones $\left[ x\left( t\right) ,y\left(
t\right) ,z\left( t\right) \right] $ y semi-trayectorias positivas $\left(
\Gamma ^{+}\right) $ del sistema ser\'{a}n positivamente invariantes y para
condici\'{o}n inicial no negativa $\left[ x\left( 0\right) ,y\left( 0\right)
,z\left( 0\right) \geq 0\right] $ se localizar\'{a}n en el siguiente dominio:

\begin{equation*}
R_{+,0}^{3}=\{x\left( t\right) ,y\left( t\right) ,z\left( t\right) \geq 0\}
\end{equation*}

\bigskip

\section{Localizaci\'{o}n de conjuntos compactos invariantes.}

Pimero se propone una funci\'{o}n localizadora para sistemas biol\'{o}gicos
con din\'{a}mica localizada en el ortante no negativo, se suguiere explorar
las siguientes funciones:

\begin{eqnarray*}
h_{1} &=&x, \\
h_{2} &=&y, \\
h_{3} &=&z, \\
h_{4} &=&x+y+z \\
h_{5} &=&x+z \\
h_{6} &=&x+y \\
h_{7} &=&y+z
\end{eqnarray*}

Nota: con abse en la estructura del sistema se observa que las variables $%
x(t)$ y $y(t)$ tienen los siguientes l\'{\i}mites inferiores y superiores:

\begin{eqnarray*}
0 &\leq &x(t)\leq 1, \\
0 &\leq &y(t)\leq 1,
\end{eqnarray*}

esto corresponde con la ley de crecimiento log\'{\i}stico (crecimiento de
tipo sigmoidal), que tiende a cero al menos infinito y a uno hacia el
infinito.

\ Se explora la siguiente funcion localizadora:

\begin{equation*}
h_{1}=x,
\end{equation*}

y se calcula su derivada de Lie (derivada temporal o derivada impl\'{\i}cita
con respecto al tiempo):

\begin{equation*}
L_{f}h_{1}=\frac{dx}{dt}=\dot{x}=r_{1}x(1-b_{1}x)-a_{12}xy-a_{13}xz,
\end{equation*}

con lo cual, se formula el conjunto $S(h_{1})=\{L_{f}h_{1}=0\}$, es decir ,

\begin{equation*}
S(h_{1})=\{r_{1}x(1-b_{1}x)-a_{12}xy-a_{13}xz=0,\}
\end{equation*}

se observa que este conjunto puede reescribirse de la siguiente forma:

\begin{equation*}
S(h_{1})=\{r_{1}-r_{1}b_{1}x-a_{12}y-a_{13}z=0,\}\cup \{x=0\},
\end{equation*}

ahora, se reescribe la primera parte del conjunto, despejando la variable de
inter\'{e}s

\begin{equation*}
S(h_{1})=\{x=\frac{1}{b_{1}}-\frac{a_{12}}{r_{1}b_{1}}y-\frac{a_{13}}{%
r_{1}b_{1}}y,\}\cup \{x=0\},
\end{equation*}

con base en lo anterior se concluye lo siguiente;

\begin{equation*}
K(h_{1})=\{x_{\inf }=0\leq x(t)\leq x_{\max }=\frac{1}{b_{1}}\},
\end{equation*}

es decir, el valor m\'{\i}nimo que puede tener la soluci\'{o}n $x(t)$ es de
cero, mientras que, el valor m\'{a}ximo que puede alcanzar esta soluci\'{o}n
cuando $\ y=z=0$, es de uno (recordando que el sistema est\'{a} normalizado)

\begin{eqnarray*}
\dot{x} &=&r_{1}x(1-b_{1}x)-a_{12}xy-a_{13}xz, \\
\dot{y} &=&r_{2}y(1-b_{2}y)-a_{21}xy, \\
\dot{z} &=&(r_{3}-a_{31})xz-d_{3}z+\rho _{i,}
\end{eqnarray*}

\bigskip

La segunda funci\'{o}n localizadora:

\begin{equation*}
h_{2}=y,
\end{equation*}

se c\'{a}lcula la derivada de Lie con respecto al tiempo:

\begin{equation*}
L_{f}h_{2}=\frac{dy}{dt}=\dot{y}=r_{2}y(1-b_{2}y)-a_{21}xy,
\end{equation*}

con lo cual, se formula el conjunto $S(h_{2})=\{L_{f}h_{2}=0\}$, es decir ,

\begin{equation*}
S(h_{2})=\{y=\frac{1}{b_{2}}-\frac{a_{21}}{r_{2}b_{2}}x\}\cup \{y=0\},
\end{equation*}

se observa que este conjunto puede reescribirse de la siguiente forma:

\begin{equation*}
K(h_{2})=\{y_{\inf }=0\leq y(t)\leq y_{\max }=\frac{1}{b_{2}}\}
\end{equation*}

es decir, el valor m\'{\i}nimo que puede tener la soluci\'{o}n $y(t)$ es de
cero, mientras que, el valor m\'{a}ximo que puede alcanzar esta soluci\'{o}n
cuando $\ x=0$, es de uno (recordando que el sistema est\'{a} normalizado)

\begin{eqnarray*}
\dot{x} &=&r_{1}x(1-b_{1}x)-a_{12}xy-a_{13}xz, \\
\dot{y} &=&r_{2}y(1-b_{2}y)-a_{21}xy, \\
\dot{z} &=&(r_{3}-a_{31})xz-d_{3}z+\rho _{i,}
\end{eqnarray*}

\bigskip

\bigskip Ahora, con base en la siguiente funci\'{o}n localizadora:

\begin{equation*}
h_{3}=z_{i}
\end{equation*}

al calcular su derivada de Lie:

\begin{equation*}
L_{f}h_{3}=\left( r_{3}-a_{31}\right) xz-d_{3}z+\rho _{i,}
\end{equation*}

se obtiene el conjunto $S(h_{3})$ como se muestra a continuaci\'{o}n:

\begin{equation*}
S(h_{3})=\{L_{f}h_{3}=0\}=\{\left( r_{3}-a_{31}\right) xz-d_{3}z+\rho
_{i}=0\}
\end{equation*}

donde, al observar los valores de los par\'{a}metros, se construye la
siguiente condici\'{o}n:

\begin{equation*}
r_{3}>a_{31},
\end{equation*}

por lo tanto, se reescribe el conjunto $S\left( h_{3}\right) $ de la
siguiente forma:

\begin{equation*}
S\left( h_{3}\right) =\left\{ z=\frac{\rho _{i}}{d_{3}}+\frac{r_{3}-a_{31}}{%
d_{3}}xz\right\} ,
\end{equation*}

por lo tanto, se observa que, la soluci\'{o}n tiene el mismo l\'{\i}mite
inferior:

\begin{equation*}
K(z)=\left\{ z\left( t\right) \geq z_{\inf }=\frac{\rho _{i}}{d_{3}},\right\}
\end{equation*}

recordando que $\rho _{i}$ es el par\'{a}metro de tratamiento/terapia (o par%
\'{a}metro de control), que puede tener valores no negativos, es decir $\rho
_{i}\geq 0.$

Por lo tanto, con base en el resultado anterior, se procede a aplicar el
denominado Teorema Iterativo del m\'{e}todo de LCCI, entonces, se reescribe
el conjunto $\ \ \ \ \ \ S(h_{1})$ como se muestra a continuaci\'{o}n:

\begin{eqnarray*}
S(h_{1}) &=&\{r_{1}-r_{1}b_{1}x-a_{12}y-a_{13}z=0,\}\cup \{x=0\},\text{
(Esto no se escribe)} \\
S\left( h_{1}\right) \cap K(z) &\subset &\{x=\frac{1}{b_{1}}-\frac{a_{12}}{%
r_{1}b_{1}}y-\frac{a_{13}}{r_{1}b_{1}}z_{\inf }\},
\end{eqnarray*}

\bigskip

ahora, al descartar el t\'{e}rmino negativo de $y$, se concluye el siguiente
l\'{\i}mite superior para la variable $x\left( t\right) $:

\begin{equation*}
K_{i}=\{x_{\inf }=0\leq x\left( t\right) \leq x_{\sup }=\frac{1}{b_{1}}-%
\frac{a_{13}}{r_{1}b_{1}d_{3}}\rho _{i}\},
\end{equation*}

Finalmente, se toma la siguiente funci\'{o}n localizadora:

\begin{equation*}
h_{4}=ax+z
\end{equation*}

cuya derivada de Lie se muestra a continuaci\'{o}n:

\begin{equation*}
L_{f}h_{4}=\alpha \lbrack r_{1}x\left( 1-b_{1}x\right)
-a_{12}xy-a_{13}xz]+(r_{3}-a_{31})xz-d_{3}z+\rho _{i}
\end{equation*}

y se determina el conjunto $S\left( h_{4}\right) =\{L_{f}h_{4}=0\}$ de la
siguiente forma:

\begin{eqnarray*}
S\left( h_{4}\right) &=&\{ar_{1}x-b_{1}\alpha r_{1}x^{2}-\alpha
a_{12}xy-\alpha a_{13}xz+\left( r_{3}-a_{31}\right) xz-d_{3}z+\rho _{i}=0\}
\\
S\left( h_{4}\right) &=&\{\rho _{i}-b_{1}ar_{1}x^{2}+\alpha r_{1}x-\alpha
a_{12}xy-\left( \alpha a_{13}-r_{3}+a_{31}\right) xz-d_{3}z=0\}
\end{eqnarray*}

para asegurar que todos los t\'{e}rminos cruzados/no l\'{\i}neales/cuadr\'{a}%
ticos, sean negativos, se impone la siguiente condici\'{o}n:

\begin{eqnarray*}
\alpha a_{13}-r_{3}+a_{31} &>&0 \\
\alpha &>&\frac{r_{3}-a_{31}}{a_{13}}
\end{eqnarray*}

ahora la funci\'{o}n localizadora se puede expresar de esta forma:

\begin{equation*}
z=h_{4}-ax
\end{equation*}

para sustituir en la siguiente expresi\'{o}n:

\begin{equation*}
S\left( h_{4}\right) =\{d_{3}z=\rho _{i}-b_{1}ar_{1}x^{2}+\alpha
r_{1}x-\alpha a_{12}xy-\left( \alpha a_{13}-r_{3}+a_{31}\right) xz\}
\end{equation*}

es decir,

\begin{eqnarray*}
S\left( h_{4}\right) &=&\{d_{3}\left( h_{4}-ax\right) =\rho
_{i}-b_{1}ar_{1}x^{2}+\alpha r_{1}x-\alpha a_{12}xy-\left( \alpha
a_{13}-r_{3}+a_{31}\right) xz\} \\
S\left( h_{4}\right) &=&\{d_{3}h_{4}-d_{3}ax=\rho
_{i}-b_{1}ar_{1}x^{2}+\alpha r_{1}x-\alpha a_{12}xy-\left( \alpha
a_{13}-r_{3}+a_{31}\right) xz\} \\
S\left( h_{4}\right) &=&\{h_{4}=\frac{\rho _{i}}{d_{3}}-\frac{b_{1}ar_{1}}{%
d_{3}}+\frac{\left( ar_{1}+d_{3}a\right) }{d_{3}}x-\frac{\alpha a_{12}}{d_{3}%
}xy-\frac{\alpha a_{13}-r_{3}+a_{31}}{d_{3}}xz\},
\end{eqnarray*}

para continuar con el proceso, primero se debe completar el cuadrado con los
siguientes dos t\'{e}rminos:

\begin{equation*}
-\frac{b_{1}ar_{1}}{d_{3}}+\frac{\left( ar_{1}+d_{3}a\right) }{d_{3}}%
x=-Ax^{2}+Bx=-A\left( x-\frac{B}{2A}\right) ^{2}+\frac{B^{2}}{4A}
\end{equation*}

y se sustituye $S\left( h_{4}\right) $

\begin{equation*}
S\left( h_{4}\right) =\{h_{4}=\frac{\rho _{i}}{d_{3}}+\frac{B^{2}}{4A}%
-A\left( x-\frac{B}{2A}\right) ^{2}-\frac{\alpha a_{12}}{d_{3}}xy-\frac{%
\alpha a_{13}-r_{3}+a_{31}}{d_{3}}xz\}
\end{equation*}

por lo tanto, se concluye el siguiente l\'{\i}mite superior para la funci%
\'{o}n $h_{4}$:

\begin{equation*}
K\left( h_{4}\right) =\{ax\left( t\right) +z\left( t\right) \leq \frac{\rho
_{i}}{d_{3}}+\frac{\alpha \left( d_{3}+r_{1}\right) ^{2}}{4b_{1}d_{3}r_{1}}\}
\end{equation*}

y se aproxima el siguiente l\'{\i}mite superior para la variable $z\left(
t\right) $:

\begin{equation*}
K_{z}=\{z_{\inf }=\frac{\rho _{i}}{d_{3}}\leq z_{\sup }=\frac{\rho _{i}}{%
d_{3}}+\frac{\alpha \left( d_{3}+r_{1}\right) ^{2}}{4b_{1}d_{3}r_{1}}\}
\end{equation*}

Con base en lo mostrado en esta secci\'{o}n, se concluye el siguiente
resultado:

\bigskip

\textbf{Resultado II. Dominio de localizaci\'{o}n: }Todos los conjuntos
compactos invariantes del sistema (ref:dx) - (ref:dz) se encuentran
localizados dentro o en las fronteras del siguiente dominio de localizaci%
\'{o}n.

\begin{equation*}
K_{xyz}=K_{x}\cap K_{y}\cap K_{z}
\end{equation*}%
donde:

\begin{eqnarray*}
K_{x} &=&\{x_{\inf }=0\leq x\left( t\right) \leq x_{\sup }=\frac{1}{b_{1}}-%
\frac{a_{13}}{r_{1}b_{1}d_{3}}\rho _{i}\} \\
K_{y} &=&\{y_{\inf }=0\leq y\left( t\right) \leq y_{\sup }=\frac{1}{b_{2}}\}
\\
K_{z} &=&\{z_{\inf }=\frac{\rho _{i}}{d_{3}}\leq z\left( t\right) \leq
z_{\sup }=\frac{\rho _{i}}{d_{3}}-\frac{\alpha \left( d_{3}+r_{1}\right) ^{2}%
}{4b_{1}d_{3}r_{1}}\}
\end{eqnarray*}

\subsection{No existencia de conjuntos compactos invariantes}

A partir del resultado mostrado en el conjunto $K_{x}$; es posible
establecer lo siguiente con respecto a la existencia de conjuntos compactos
invariantes para la variable $x\left( t\right) :$

\textbf{Resultado III. No existencia. }\textit{Si la siguiente condici\'{o}n
sobre el par\'{a}metro de tratamiento/tearpia se cumple:}

\begin{equation*}
\frac{1}{b_{1}}-\frac{a_{13}}{r_{1}b_{1}d_{3}}\rho i\leq 0
\end{equation*}

\textit{es decir,}

\begin{equation*}
\rho i\geq \frac{r_{1}d_{3}}{a_{13}}
\end{equation*}

\textit{entonces, se puede asegurar la no existencia de conjuntos compactos
invariantes fuera del plano }$\ x=0$\textit{, por lo tanto cualquier din\'{a}%
mica que pueda exhibir el sistema, estar\'{a} localizada dentro o en las
fronteras del siguiente dominio:}

\begin{equation*}
K_{yz}=\{x=0\}\cap K_{y}\cap K_{z}
\end{equation*}

\subsection{Puntos de equilibrio}

Para calcular los puntos de equilibrio del sistema (ref: dx)-(ref-dz), se
igualan a cero las ecuaciones como se muestra a continuaci\'{o}n:

$\func{assume}\left( r_{1},\func{positive}\right) =\allowbreak \left(
0,\infty \right) $

$\func{assume}\left( b_{1},\func{positive}\right) =\allowbreak \left(
0,\infty \right) $

$\func{assume}\left( a_{12},\func{positive}\right) =\allowbreak \left(
0,\infty \right) $

$\func{assume}\left( a_{13},\func{positive}\right) =\allowbreak \left(
0,\infty \right) $

$\func{assume}\left( r_{2},\func{positive}\right) =\allowbreak \left(
0,\infty \right) $

$\func{assume}\left( a_{21},\func{positive}\right) =\allowbreak \left(
0,\infty \right) $

$\func{assume}\left( r_{3},\func{positive}\right) =\allowbreak \left(
0,\infty \right) $

$\func{assume}\left( a_{31},\func{positive}\right) =\allowbreak \left(
0,\infty \right) $

$\func{assume}\left( d_{3},\func{positive}\right) =\allowbreak \left(
0,\infty \right) $

Primero, se calculan los equilibrios asumiendo $\rho _{i}=0$:

\begin{eqnarray*}
0 &=&r_{1}x\left( 1-b_{1}x\right) -a_{12}xy-a_{13}xz \\
0 &=&r_{2}y\left( 1-b_{1}y\right) -a_{21}xy \\
0 &=&(r_{3}-a_{31})xz-d_{3}z+\rho _{i}
\end{eqnarray*}

\section{Condiciones de eliminaci\'{o}n}

Las condiciones de eliminaci\'{o}n se establecen sobre el par\'{a}metro de
tratamiento/terapia o control, y se determinan al aplicar la teor\'{\i}a de
estabilidad en el sentido de Lyapunov, part\'{\i}cularmente el m\'{e}todo
directo de Lyapunov.

Se propone la siguiente funci\'{o}n candidata de Lyapunov:

\begin{equation*}
V=x
\end{equation*}

y se calcula su derivada

\begin{equation*}
\dot{V}=\dot{x}=r_{1}x\left( 1-b_{1}x\right) -a_{12}xy-a_{13}xz
\end{equation*}

se reescribe la derivada de la siguiente forma:

\begin{equation*}
\dot{V}=\left( r_{1}-r_{1}b_{1}x-a_{12}y-a_{13}z\right) x
\end{equation*}

ahora, al considerar los resultados del dominio de localizaci\'{o}n y
evaluar la derivada en este, es decir,

\begin{equation*}
\left. \dot{V}\right\vert _{K_{xyz}}
\end{equation*}

se tiene lo siguiente

\begin{equation*}
\dot{V}=\left( r_{1}-a_{13}z_{\inf }\right) x\leq 0
\end{equation*}

a partir de esta expresi\'{o}n, se establece la siguiente condici\'{o}n:

\begin{equation*}
-a_{13}\frac{\rho _{i}}{d_{3}}<0
\end{equation*}

por lo tanto, se despeja el par\'{a}metro de tratamiento/terapia o control:

\begin{equation*}
\rho _{i}>\frac{d_{3}r_{1}}{a_{13}}
\end{equation*}

y se establece el siguiente resultado:

\bigskip

\textbf{Resultado IV: Condiciones de eliminaci\'{o}n. }\textit{Si la
siguiente condici\'{o}n se cumple:}

\begin{equation*}
\rho _{i}>\frac{d_{3}r_{1}}{a_{13}}
\end{equation*}

\textit{entonces, se puede asegurar la eliminaci\'{o}n de la poblaci\'{o}n
descrita por la variable }$x\left( t\right) $, es decir

\begin{equation*}
\lim_{t\rightarrow \infty }x\left( t\right) =0
\end{equation*}

\bigskip

\bigskip

Referencia:

\begin{enumerate}
\item D.Leenheer,P., \& Aeyels,D. (2001). Stability propierties of
equilibria of classes of cooperative systems. IEEE Transactions on Automatic
Control, 46(12), 1996-2001. https://doi.org/10.1109/9.975508
\end{enumerate}

\end{document}
